\graphicspath{{chapters/10/images/}}
\chapter{Bioreactors}

\section{Scaffold development process}
The starting point for scaffold development is the study of a pathology.
Then the microenvironemnt, composed of chemical and mechanical signal, the ability to heal and the tissue to regenerate are analysed.
From this informations a scaffold able to mimic the original biological morphology can be built.
The goal must be reached by carefully designing materials and manufacturing methods.
For example 3D printing layer by layer allows us to build gradients with different properties.
After obtaining a scaffold, it needs to be characterized with its biological function.
Once a good result is obtained clinical trials can begin to asses the absence of damages to the body and to demonstrate the suitability for therapeutic purposes.

	\subsection{3D structures}
	Growing cells on flat surfaces is unnatural and artificial and it does not make sense in a biological perspective.
	Natural ECM plays an important role in regulating cellular behaviours by influencing cells with biochemical signals and topographical cues.
	In 3D cultures scaffold morphology, architecture and chemistry, bio recognition signaling, degradation mechanisms and cell patterns can be controlled.
	Cells behave and respond to stimuli more like they would do in vivo in a 3D environment.
	Moreover pharmaceutical industries are looking for trustful models with the purpose of evaluating the impact of a drug.

		\subsubsection{Advantages over 2D structures}
		2D culture substrates are not able to reproduce the complex and dynamic environments of the body, forcing cells to adjust to an artificial flat, rigid surface.
		3D matrices or scaffolds are porous substrates that can support cell growth, organisation, and differentiation on or within their structure.
		Architectural and material diversity have much more impact on 3D matrices than on 2D substrates.
		Chemical and biochemical modification with specific biological motives can be added to facilitate cell adhesion, cell mediated proteolytic degradation and growth factor binding and release.

	\subsection{Focus on biological processes}
	In order to build an effective scaffold high level of precision are necessary.
	Moreover there is a need to stick to biological processes.
	While working in vitro, we must carefully choose cells, like monocytes, to evaluate immune response.
	The mechanics should be dynamic, not static.
	All the stresses to apply, their intensity and timing must be decided for an in vitro evaluation.
	Bioreactors should always be designed by keeping in mind the application.

	\subsection{Biomimetic approach}
	The starting point in engineering matrices is a reliable biological model.
	Matrices contain some factors that are able to induce specific cellular behaviours like speed up regeneration or drive the healing process.
	The best way to engineer a tissue is through the biomimetic approach: mimic the main aspects of the biological material by isolating specific signals.

\section{3D culture engineering}
To check the goodness of a scaffold it can be cultured together with stem cells in a bioreactor and check if they differentiate into the desired phenotype

	\subsection{Challenges of 3D culture engineering}
	The challenges to overcome in 3D culture engineering are:

	\begin{multicols}{2}
		\begin{itemize}
			\item Assess the number of cells necessary to replace the normal function.
			\item In vitro tests must be carefully designed to recapitulate nutritional requirements, structural needs, mechanical support and shear stress>
			\item Design a large scale industry product.
				Complexity, time for production and cost must be taken into account.
				Reproducibility must be guaranteed, as well as sterilization and eventual parameters that can be modified.
		\end{itemize}
	\end{multicols}

	\subsection{In vitro testing}
	Before moving to animal models, in vitro the cell or tissue responses should be checked.
	To do so different instruments, among which imaging is the most popular, can be used.
	Furthermore immune staining for molecules, receptors or other molecules can be performed.

		\subsubsection{Mechanical stress}
		Matrix elasticity and stiffness are very relevant in inducing specific behaviours in stem cells.
		Adipocytes are the cells that require the lowest stiffness, while bone cells the highest.
		Some example of what forces impact what tissue:

		\begin{multicols}{2}
			\begin{itemize}
				\item Shear stress: endothelium: blood flow.
				\item Compression: cartilage.
				\item Hydrostatic pressure: chrondrcytes.
				\item Tensions: tendons.
			\end{itemize}
		\end{multicols}

		In all cases we also need to define the strength, duration and timing of the stress to be applied.

		\subsubsection{Morphology}
		The morphology of the scaffold could modulate the phenotype of the cells and the function.
		Some possible morphologies include:

		\begin{multicols}{2}
			\begin{itemize}
				\item Roughness: modulation in phenotype of osteoblasts in permanent prosthasis
				\item Random features.
				\item Anisotropic features.
				\item Isotropic features.
			\end{itemize}
		\end{multicols}

	\subsection{Case of study - cartilage}
	Soluble signalling molecules in cartilage collagenic structures differ: there is an intermediate transition zone and bone.
	Cartilage regeneration is difficult due to lack of vascularization but can be induced by injecting adipose derived stem cells into microspheres.
	Two types of microspheres should be employed, as two different signals are required to control timing.
	In this case the patient is the bioreactor.
	The polymer is sensible to UV light for polymerization.
	The irradiation should be safe for cells, we need to require a specific intensity.
	Furthermore, the irradiation should be really focused on the polymeric matrix.

	\subsection{Mathematical modelling of 3D cultures}
	We need to define computational approaches and in silico models to carefully evaluate parameters.
	Furthermore, cell growth and migration can be modelled.

\section{Bioreactors}
Bioreactors need to be changed according to specific applications like, for example, 3d print scaffold directly inside the bioreactor.
A bioreactor is composed by a chamber containing cells and an in and out circulation system to remove and add culture medium.
New ones can have sensors for level of oxygen, the production of specific molecules and others application specific sensors.
Mechanical stresses are induced like medium circulation or rotation to create shear stress movement.
Bioreactors can be used for large scale cell cultivation like genetically modified bacteria for cellulose production.
A bioreactor should provide efficient mass transfer through the scaffold, precise spatial distribution of cells in 3D and expose tissue to physical stimuli.
The bioreactor is responsible for the environment control, nutrient delivery and mechanical stimulation.
