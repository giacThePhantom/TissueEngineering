\graphicspath{{chapters/10/images/}}
\chapter{Bioreactors}

\section{Scaffold development process}
We start from a pathology, analyze the microenvironment (not only chemical signals, but also mechanical signals and the ability to heal) and the tissue to regenerate. 
From this info we can build a scaffold able to mimic the original biological morphology. 
The goal must be reached by carefully designing materials and manufacturing methods. 
E.g. 3D printing layer by layer allows us to build gradients with different properties. 
After obtaining a scaffold, we need to characterize it with biological function. 
Once a good result is obtained ,we can pass to clinical trials: first of all we need to make sure that we are not producing damages to the body, secondly it is required to demonstrate the suitability for therapeutic purposes.

\section{3D structures}
Growing cells on flat surfaces is unnatural and artificial, it does not make sense in a biological perspective.
Natural ECM plays an important role in regulating cellular behaviours by influencing cells with biochemical signals and topographical cues.
In 3D cultures, we can control scaffold morphology, architecture and chemistry, bio recognition signaling, degradation mechanisms, patterns; cells behave and respond to stimuli more like they would do in vivo.
\\
\\
\noindent
2D culture substrates are not able to reproduce the complex and dynamic environments of the body, forcing cells to adjust to an artificial flat, rigid surface. 
3D matrices or scaffolds are porous substrates that can support cell growth, organisation, and differentiation on or within their structure. 
Architectural and material diversity have much more impact on 3D matrices than on 2D substrates. 
Chemical and biochemical modification with specific biological motives can be added to facilitate cell adhesion, cell mediated proteolytic degradation and growth factor binding and release. 
\\
\\
\noindent
In order to build an effective scaffold we must be very precise and stick to biological processes.  
While working in vitro, we must carefully choose cells e.g. monocytes to evaluate immune response. 
The mechanics should be dynamic, not static. 
We need to decide stresses to apply, their intensity and timing. 
Bioreactors should always be designed by keeping in mind the application.

\section{Biomimetic approach}
In order to engineer matrices, we need to start from a reliable biological model. 
Matrices contain some factors that are able to induce specific cellular behaviours e.g. speed up regeneration, drive the healing process. 
The best way to engineer a tissue is through the biomimetic approach i.e. mimic the main aspects of the biological material by isolating specific signals. 
\\
\\
\noindent
Pharmaceutical industries are looking for trustful models with the purpose of evaluating the impact of a drug. 
3D tissue development allows us to exploit cancer tissues e.g. high patient variability in cancer can be overcome by building a patient-specific model. 

\section{3D culture engineering}
We start from stem cells and a scaffold, culture in bioreactor and check whether stem cells differentiate in the desired phenotype.  
\\
\\
\noindent
Challenges:
\begin{itemize}
\item how many cells are needed to replace normal function?
\item in vitro tests must be carefully designed to recapitulate nutritional requirements, structural needs, mechanical support and shear stress
\item Design a large scale industry product. We need to take into account complexity, time for production and cost. Reproducibility must be guaranteed, as well as sterilization and eventual parameters that can be modified.
\end{itemize}

\section{In vitro testing}
Before moving to animal models, we should check in vitro the cell/tissue responses.
We can use different instruments, among which imaging is the most popular. 
Furthermore, we can perform immune staining for molecules, receptors, etc.

\subsection{Mechanical stress}
Matrix elasticity/stiffness is very relevant in inducing specific behaviours in stem cells. 
The lowest stiffness is required for adipocytes, highest for bone. 
\begin{itemize}
\item shear stress e.g. endothelium: blood flow
\item compression e.g. cartilage
\item hydrostatic pressure e.g. chrondrcytes
\item tensions e.g. tendons
\end{itemize}
In all cases we also need to define the strength, time etc.

\subsection{Morphology}
\begin{itemize}
\item roughness e.g. modulation in phenotype of osteoblasts in permanent prosthasis
\item random features
\item anisotropic features
\item isotropic features
\end{itemize}

\section{Case of study}
Soluble signalling molecules in cartilage collagenic structures differ,  we have  an intermediate transition zone and bone. 
We can induce cartilage regeneration (difficult due to lack of vascularization) by injecting adipose derived stem cells into microspheres. Two types of microspheres should be employed,  as two different signals are required to control timing. 
In this case the patient is the bioreactor.
The polymer is sensible to UV light for polymerization. 
The irradiation should be safe for cells, we need to require a specific intensity. 
Furthermore, the irradiation should be really focused on the polymeric matrix. 

\section{Mathematical modelling of 3D cultures}
We need to define computational approaches and in silico models to carefully evaluate parameters. Furthermore, cell growth and migration can be modelled.

\section{Bioreactors}
Bioreactors need to be changed according to specific applications e.g. 3d print scaffold directly inside the bioreactor. 
A bioreactor is composed by a chamber containing cells and an in and out circulation system (to remove and add culture medium).
New ones can have sensors for level of oxygen, the production of specific molecules etc.
Mechanical stresses are induced e.g. medium circulation or rotation(shear stress movement).
Bioreactors can be used for large scale cell cultivation e.g. genetically modified bacteria for cellulose production.  A bioreactor should provide efficient mass transfer through the scaffold, precise spatial distribution of cells in 3D and expose tissue to physical stimuli. 
The bioreactor is responsible for the environment control, nutrient delivery and mechanical stimulation.





