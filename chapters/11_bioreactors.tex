\graphicspath{{chapters/bioreactors/}}
\chapter{Bioreactors}

\section{Scaffold development process}
We start from a pathology, analyze the microenvironment (not only chemical signals but also mechanical signals, ability to heal) and the tissue to regenerate. From this info we can build scaffold able to mimic the morphology. The goal must be reached by carefully designing materials and manufacturing methods. E.g. 3D printing layer by layer allows us to obtain gradients with different properties. After obtaining a scaffold we need to characterize it with biological function. Once a good result is obtained we can pass to clinical trials: first of all make sure that we are not producing damages to the body, secondly demonstrate suitability  for therapeutic purposes.

\section{3D structures}
Growing cells on flat surfaces is unnatural and artificial, it does not make sense in a biological perspective.
Natural ECM plays an important role in regulating cellular behaviours by influencing cells with biochemical signals and topographical cues.
In 3D cultures, we can control scaffold morphology, architecture and chemistry, bio recognition signaling, degradation mechanisms, patterns; cells behave and respond to stimuli more like they would do in vivo.
\\
\\
\noindent
2D culture substrates are not able to reproduce the complex and dynamic environments of the body, forcing cells to adjust to an artificial flat, rigid surface. 3D matrices or scaffolds are porous substrates that can support cell growth, organisation, and differentiation on or within their structure. Architectural and material diversity have much more impact on 3D matrices than on 2D substrates. Other than physical properties, chemical and biochemical modification with specific biological motives to facilitate cell adhesion, cell mediated proteolytic degradation and growth factor binding and release.
\\
\\
\noindent
In order to build an effective scaffold we must be very precise and stick to biological processes.  While working in vitro we must carefully choose cells e.g. monocytes to evaluate immune response. The mechanics should be dynamic, not static. We need to decide stresses to apply, their intensity and timing. Bioreactors should always be designed by keeping in mind the application.
