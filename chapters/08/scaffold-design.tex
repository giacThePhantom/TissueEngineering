\graphicspath{{chapters/08/images/}}
\chapter{Scaffold design}

The dynamics of regeneration vary from tissue to tissue according to the hierarchy of tissue or organ function.
Remember that “the ability of a material to perform with an appropriate host response in a specific application”.
Important consideration for the materials:
\begin{itemize}
\item sourcing of functional cells: if the scaffold requires a pre-loading of cells, we need to discuss which kind of cells to employ beforehand e.g. stem cells, cells from the patient (bone marrow, amniotic liquid, etc)
\item GF regulatory systems: synthetic polymers with no biorecognition properties require functionalization
\item immuno acceptance: e.g. force stem cells to become osteoblasts or modulate the immune response in order to shorten the inflammation and boost regeneration
\end{itemize}
\noindent
Biological information is required for performing in vitro tests and for functionalization (linked to specific biological pathways, essential to thoroughly describe the mechanisms).
\\
\\
\noindent
Scaffolds can be grouped in two categories:
\begin{itemize}
\item conductive: provide and maintain a 3D environment that supports a passive cell infiltration, creating a pseudo micro environment. The limitation is that they are not able to provide enough info to promote full regeneration in most applications.
\item  inductive: designed toclosely mimic the native cellular environment and may contain bioactive molecules and naturally or synthetic analogues of structural, functional or specialised proteins and proteoglycans. They can increase the biocomplexity of the system.
\end{itemize}
\noindent
A scaffold is part of a complex system. It is used to direct, by control of interactions with components of living systems, the course of any therapeutic or diagnostic procedure.
A scaffold is composed by different materials e.g. natural and synthetic polymers.
The geometry should be suitable for the specific application e.g. allow migration, cell distribution in 3D in functional manner, alignment.
Polymers can be combined with other materials or drugs.
If the interaction and biocompatibility are present, we will have a suitable response.

\subsection{Tissue engineering strategies}
\begin{itemize}
\item "just" scaffold in vivo: the body helps with the regeneration, if possible it’s the best strategy we can follow.
\item scaffold + cells implantation: we need to define why we wish to culture in vivo e.g. ECM production, differentiation,…
\item cell sheet engineering: fabrication with cells (stem cells) from the patient.
\end{itemize}
\noindent
Cell sheet engineering is the only bottom-up approach, starting from the material and not the biology.
%Thermo responsive polymers: different orientations and activity depending on the temperature while lamina: thermo responsive polymers, culture at 37° C → laminar produced after this procedure.

\subsection{Biocompatibility requirements}
Revised performance criteria for the 4th generation of biomaterials:
\begin{itemize}
\item tailored biodegradation
\item amenability to engineering design and manufacturing
\item induces cell and tissue integration
\item smart (i.e. physiologically responsive)
\item instructional (i.e. controls cell fate)
\item mechanical strength and function (i.e. mechanical signalling)
\end{itemize}

\begin{figure}[h]
\centering
\includegraphics[width=0.5\textwidth]{sponge}
\caption{\label{fig:sponge}}
\end{figure}
\noindent
Fig \ref{fig:sponge} repair of trabecular bone with 3D-scaffold sponges.
All of them are artificial scaffolds, while on the top right we have the natural sponge.
The porosity can be oriented or random, with different geometries.
The scaffold should promote adhesion, proliferation and migration. Hypoxia should be avoided, maybe with early angiogenesis.
In vivo: in the case of bones, we have osteoblasts adhesion and migration, ECM production. In order to avoid hypoxia and necrotic tissue formation it is required to achieve vascularization, we need to supply oxygen to stimulate angiogenesis.
In vitro: millimetres not microns, the problem is more tough. We could culture in a bioreactor e.g. chamber with medium forced to go through, or include into the scaffold oxygen donors.

\subsection{Experimental data discussion}
\begin{figure}[h]
\centering
\includegraphics[width=0.5\textwidth]{myoblast.jpg}
\caption{\label{fig:myoblast}}
\end{figure}
\noindent
Fig \ref{fig:myoblast}: muscle regeneration. Use collagen, elastin and glycosaminoglycans. GAGs control water content and mechanical properties. Elastin is required for muscle elasticity, collagen for the strength.
It is necessary to find the optimal ratio among the components.
By changing the ratio,  we have 4 different architectures:
\begin{enumerate}
\item C: sponge, similar to natural behaviour of collagen
\item C/E 9:1: the big fiber starts to appear (elastin)
\item C/E 1:1: similar content
\item C/E 1:9: a lot of big fibers pink fiber: elastin fiber
\end{enumerate}
\noindent
C might not be optimal, as well as C/E 1:9, as they are quite different from biological setting.
In vitro test: after two weeks myoblasts were cultured for 14 days and myotubes were formed on all scaffolds except C/E 1:9(D).  We witness a different organisation of the myotubes; in the last example, myoblasts are not able to organise as the condition is very far from physiology.
By considering the test, the best samples seem to be C/E 9:1 and C/E 1:1 .
\\
\\
\noindent
Tissue engineering is 3D assembly over time of vital tissues-organs by a process involving cells, signals, and the extracellular matrix.
\begin{figure}[h]
\centering
\includegraphics[width=0.6\textwidth]{3D.png}
\caption{\label{fig:3D}}
\end{figure}
In this case we have a scaffold for bone regeneration and osteoblast proliferation (fig \ref{fig:3D}).  The cells (green) are able to adhere, proliferate and migrate. The scaffold is then completely covered in cells. We then witness tissue-specific ECM production, mineralzation. This is a great starting point, but we have to better understand the degradation process of the scaffold and also vascularisation!
How can we check in vivo vascularisation? We need to perform co-culture of osteoblasts (angiogenesis factors + collagenic structure for capillary network formation) and endothelial cells (use empty spaces to assemble as a tube). To obtain a physiological situation we need a careful design.

\section{Scaffold examples}
\subsection{Example 1}
\begin{figure}[h]
\centering
\includegraphics[width=0.6\textwidth]{sk_car.jpg}
\caption{\label{fig:sk_car}}
\end{figure}
Fig \ref{fig:sk_car}: film with same polymers, same architecture, different cells: keratinocytes and chondrocytes. The two films were completely spread.
Cells grown in a flat dish tend to behave as individual cells or forming a monolayer, whereas cells cultured in a 3D space are more likely to assume the characteristics of a particular tissue.
Cartilage, once grown flat, is almost impossible to shape into joints. The scaffold here, in the case of cartilage, is inducing the loss of the original phenotype,  so it is not biocompatible.
In the case of the skin, it is promising for biocompatibility, very compact and well connected monolayers.

\subsection{Example 2}
\begin{figure}[h]
\centering
\includegraphics[width=0.5\textwidth]{bone.jpg}
\caption{\label{fig:bone}}
\end{figure}
Fig \ref{fig:bone}: in this case we do not have biocompatibility at all. The cells adhere to the polymer but do not migrate, so we should increase the porosity.
After few days: we have a multi-layer as result, no migration and no biocompatibility.

\subsection{Example 3}
\begin{figure}[h]
\centering
\includegraphics[width=0.5\textwidth]{suine.jpg}
\caption{\label{fig:suine}}
\end{figure}
Fig \ref{fig:suine} suine primary urothelial cells, 6 days after seeding.
The right image has the geometrical shape of the cells, the phenotype is ok.
Left: cells don’t connect and are disorganised, so they do not communicate.
They are more round shaped, so the adherence is not good, there is no activation.

\subsection{Example 4}
\begin{figure}[h]
\centering
\includegraphics[width=0.5\textwidth]{fibroin}
\caption{\label{fig:fibroin}}
\end{figure}
Fibroin micronet (Fig \ref{fig:fibroin}) human microcapillary endothelial cells (HDMEC) and primary human osteoblast cells (HOS) in coculture (10 days). Very good results, empty tubes, but it is promising. They can be put into the scaffold and then maybe used as vessels. Issue: We have to pay attention if the vessels connect to our circulatory system!

\subsection{Example 5}
\begin{figure}[h]
\centering
\includegraphics[width=0.5\textwidth]{osteoblast}
\caption{\label{fig:osteoblast}}
\end{figure}
Figure \ref{fig:osteoblast}: pre-vascularized fibroin net, in vivo anastomosis with host vasculature.
Brown tubes: in vivo, blood cells are present and anastomise.
Left: loaded with osteoblast, let grow and then implanted. Fast angiogenesis, not too many vessels, only in the surface
Right: loaded with osteoblast, immediately implanted.
In vitro human cells and implanted in rat to identify the difference. On the right there are some capillaries that were produced and there is not red blood cells.

\subsection{Example 6}
\begin{figure}[h]
\centering
\includegraphics[width=0.5\textwidth]{salt}
\caption{\label{fig:salt}}
\end{figure}
Figure \ref{fig:salt}: from a NaCl solution salt crystals will form a sponge. The porosity depends on crystal size. We can then apply gamma rays treatment and add either silk fibroin or P(d,l)LA.


    \subsection{Ectopic implant in rat}
    An ectopic implant is done in a sit that is not natural, in the studied case under the skin for bone.

    \begin{figure}[H]
        \centering
        \includegraphics[width=0.5\textwidth]{implant_fibroin.png}
    \caption{\label{fig:implant_fibroin} \textbf{1. Fibroin.} In the \textbf{control} there is no red sign at all so for this reason we can say that there is not induction. There are cells (blue) but they are not infiltrated in the scaffold. There are no sign of mineralization. No osteoinduction. In \textbf{DPSC} and \textbf{AFSC} there are red nodules and that means that it becomes osteoinductive and force stem cells to become osteoblast and also mineralization. In the H $\&$ E column we can notice a regeneration framework. \textbf{2. PDLLA} In the \textbf{control} there is not red sign so there is no induction. There are cells (blue) but are not filtered in the scaffold. There is no mineralization or osteoinduction. In \textbf{DPSC} and \textbf{AFSC}, even in the case of pre-loaded cells there are not red signs and also there is not osteoinduction. In the center there are not cells and that suggests us that the cells started to go into apoptosis.}
    \end{figure}

    \begin{figure}[H]
        \centering
        \includegraphics[width=0.5\textwidth]{fibroin_confocal.png}
    \caption{\label{fig:fibroin_confocal} In order to assess which kind of cell is working, pre-loaded cells or osteo cells, we can do an anti-human staining.}
    \end{figure}

    \begin{figure}[H]
        \centering
        \includegraphics[width=0.5\textwidth]{intraosseous_implants.png}
    \caption{\label{fig:intraosseous_implants} \textbf{Intaosseous implants.} The \textbf{fibroin} enhance the new bone formation and supports regeneration. In \textbf{P(d,l)LA} there are lots of holes and the regenerations is not completed. It is possible to evaluate also the quality of the bone and we can see in the fibroin it was good. There is a mixture of human and murine cells that are working together.}
    \end{figure}

    \begin{figure}[H]
        \centering
        \includegraphics[width=0.5\textwidth]{mineralization.png}
    \caption{\label{fig:mineralization} \textbf{Areas of mineralization.} In the figure it is possible to see the areas of mineralization on fibroin scaffolds after 4 weeks implantation. It was calculated on 5 transversal sections cut at interval of 1 mm in the mid region of different fibroin stained by Alzarin Red. There is no mineralization in the scaffold with only fibroin, while there is mineralization in the two scaffold that present also AFSC and DPSC. There is no mineralization in the three scaffolds with PdILA.}
    \end{figure}

\section{Cell sheet engineering}

Cells sheet engineering is based on thermo-responsive polymers. They can be used in different sites like oral mucosa that present a very difficult regeneration, in periodontal ligament regeneration, but also in myocardium regeneration (as an alternative to heart transplantation).
By transplanting single cell sheets directly to host tissues, skin, cornea, periodontal ligament, and bladder can be reconstructed. Additionally, the creation of co-cultured cell sheets from dishes with dual temperature-responsive domains, also allows for the re-creation of higher-order structures such as the kidney and liver.

\begin{figure}[H]
        \centering
        \includegraphics[width=0.5\textwidth]{cells_sheet1.png}
    \caption{\label{fig:cells_sheet1} Cell sheet harvest deposited ECM (green), as well as membrane proteins, so that confluent, monolayer cells are harvested as single cells (upper right). The temperature-responsive polymer (orange) covalently immobilized on the dish surface hydrates when the temperature is reduced, decreasing the interaction with deposited ECM. All the cells connected via cell-cell junction proteins are harvested as a single, contiguous cell sheet without the need for proteolytic enzymes.}
\end{figure}

\begin{figure}[H]
        \centering
        \includegraphics[width=0.5\textwidth]{cells_sheet2.png}
        \caption{\label{fig:cells_sheet2} Atomic force microscope images of temperature-responsive culture dish surfaces. Nongrafted, polystyrene culture dish surfaces (left) and poly(N-isopropylacrylamide)-grafted culture dish surfaces (right) were examined in air}
\end{figure}

\begin{figure}[H]
        \centering
        \includegraphics[width=0.5\textwidth]{cardiac_sheet.png}
        \caption{\label{fig:cardiac_sheet} Cardiac Tissue Reconstruction Based on Cell Sheet Engineering}
\end{figure}

\begin{figure}[H]
        \centering
        \includegraphics[width=0.5\textwidth]{cell_bank.png}
        \caption{\label{fig:cell_bank} Burn skin}
\end{figure}

    \subsection{Current challenges and strategies}

    \begin{figure}[H]
        \begin{multicols}{2}
            \includegraphics[width=0.5\textwidth]{strategies1.png}
            \includegraphics[width=0.5\textwidth]{strategies2.png}
            \includegraphics[width=0.5\textwidth]{strategies3.png}
            \includegraphics[width=0.5\textwidth]{strategies4.png}
        \end{multicols}
        \caption{\label{fig:strategies} Current challenges and strategies in cells sheet engineering.}
\end{figure}

\section{Cell encapsulation}

The cell encapsulation methods consists in the entrapment of cells in microcapsules or microbeads starting from a suspension of cells in polymeric solution that can be solidified by chemical or physical methods. During the encapsulation the polymer cross-links by chemicals, magnetic field, light or by any other crosslinkers.

\begin{figure}[H]
        \centering
        \includegraphics[width=0.5\textwidth]{encapsulation.png}
        \caption{\label{fig:encapsulation} (a) dropping the polyelectrolyte solution into a solution of small ions; (b) via a water in oil emulsification technique; and (c) complexation of oppositely charged polyelectrolytes by mixing, with additional coating procedures}.
\end{figure}

\section{Electro-hydrodynamic jetting (EHDJ)}

In the electro-hydrodynamic systems a solution is fed through a positively charged metallic needle. The solution reacts to the presence of the charge, generating repulsive coulombic forces on its surfaces, causing the deformation of the meniscus at the tip of the needle into a Taylor cone. If the voltage is high enough, the electrostatic repulsion on the surface can overcome the surface tension at the apex of the liquid cone, leading to its disintegration (Rayleigh limit) and creating a jet of drops.
There are some parameters that we have to assess:

\begin{multicols}{2}
    \begin{itemize}
        \item Speed;
        \item Starting concentration of cells;
        \item Starting concentration of polymer;
        \item Type of polymer.
    \end{itemize}
\end{multicols}

Taking in consideration these parameters we are able to control the final number outside of cells onto the beads. We can also check the quality of the concentration using a confocal microscope with live/dead staining \ref{fig:staining}. With this process is possible to check how many cells are alive and assess an optimal number of cells.

    \begin{figure}[H]
        \centering
        \includegraphics[width=0.5\textwidth]{staining.png}
        \caption{\label{fig:staining}}.
\end{figure}

    \subsection{Cell encapuslation requirements}

    \begin{multicols}{2}
        \begin{itemize}
            \item The encapsulation material (polymer should permit the free passage of nutrients and oxygen (in) and waste products (out) as well as of therapeutic protein products.
            \item Encapsulation material should prevent high Mw molecules, antibodies and other immunogenic moieties from contacting the encapsulated cells.
            \item The encapsulation material should protect cells from mechanical stresses and from the host's immune-response.
            \item The encapsulation material and method should not damage cells neither affect their behavior, as related to the desired function/application.
        \end{itemize}
    \end{multicols}

    \subsection{Cell encapuslation applications}
    Possible applications for cells encapsulation are drug delivery, bioartifical organs and tissue engineering.

    \begin{multicols}{2}
        \begin{itemize}
            \item The encapsulation of therapeutic cells permits the implantation of allogenic and xenogenic cells for the regulation of certain physiological processes damaged by death or senescence of host tissues.
            \item Microcapsules injected at the transplantation bed allow the release of biomolecules produced by the encapsulated cells, such as insulin produced by encapsulated beta-cells for diabetes type I therapy, pro-angiogenic or anti-angiogenic growth factors to enhance or inhibit vascularization.
            \item Microencapsulated cells can be used as building blocks for the fabrication of tissues/tissue precursor in vitro or implanted in vivo for tissue regeneration.
        \end{itemize}
    \end{multicols}

\section{Organ Printing - Bioprinting}

\section{Short-term encapsulation effect}

\begin{figure}[H]
        \centering
        \includegraphics[width=0.5\textwidth]{short_term.png}
        \caption{\label{fig:short_term}}
\end{figure}

\begin{multicols}{2}
    \begin{itemize}
        \item The main function of HSP proteins consist in cells protection against apoptosis, necrosis, hypoxia or any other type of stress.
        \item Elevated co-erexpression of HSP70B' and caspase-3 genes points to mild stress conditions, and cells protection activity.
        \item The results were supported with Live/Dead test, where we can observed cell death within 48 hours after encapsulation
    \end{itemize}
\end{multicols}
