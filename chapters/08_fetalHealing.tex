\graphicspath{{chapters/scarring/}}
\chapter{Fetal healing}

\section{Scarring}
A scar is a densely packed disorganized collagen bundle, with absence of hair follicles, sebaceous glands and other appendages. 
Scarring and fibrosis dominate some diseases in every branch of medicine and surgery.
Examples: skin incisions heal with scars (pathological processes as keloids, hypertrophic scars, …).
\\
\\
\noindent
Strictures occur at anastomotic sites, blood vessels, trachea, ureter, bile, duct, … 
The fetus is able to regenerate tissues by assembling collagen fiber in a well organised structure.
The biology of fetal repair must be understood in particular cellular and matrix events may provide insights to help to modulate adult wound repair to become more fetal-like.
TE: scaffold/FBRx/healing process. Immune conjugate in the system. We want to copy the fetus.

What type of ECM protein should we use to initiate the process?
Scaffold based on collagen: if cells don't’ usually reside in an ECM made extensively of coll I, the cellular response during tissue regeneration will probably be pathologic! 
Example: chondrocytes usually interact with coll II, if coll I cells form fibrocartilage. 
Collagen represents a mature scaffold, forming a microenvironment suitable for fully differentiated cells.
Embryonic development: ECM is elaborated in parallel with cell differentiation and growth not normal ECM analogue (mature scaffold) but wound bed matrix analogue (immature scaffold for regeneration).

ECM will have a high content of collagen type 2 and 5, low TGF beta, more fibronectin and tenascin, HA major component and non sulfonated GAGs.

CELLS will be fetal fibroblasts, with high migration rate (affects collagen depth and cross linking)
and increased number of HA receptors. Myofibroblasts appear early than in the adult and then disappear. There are no inflammatory effector cells, platelets less degranulation of active cytokines. We cannot avoid the scar 100\% , but we can reduce the scar tissue.

A lesson from fetal healing: scarless regeneration embryonic vs adult embryonic:

\subsection{Embryonic healing}
Minimal inflammatory response, there is no hemostatic phase and low level of TGFs. Cell migration is rapid and there is no myofibroblast formation. The cytoskeleton is randomly organized and collagen is deposisted in normal basket weave architecture.  Final tissue regeneration is obtained rapidly and without reorganization.

\subsection{Adult healing}
The inflammatory response is triggered by platelets degranulation and pro fibrotic and pro inflammatory mediators ad TGF beta1 and PDGF. Following the signal, neutrophils, monocytes, macrophages, lymphocytes reach the inflammation site. Myofibroblast differentiate and excessive collagen is produced, leading to the formation of cytoskeleton stress fibers and abnormal deposition of collagen. Overall, the process results in extended wound and scar.

\subsection{Hyaluronan}
Hyaluronan is a hydrated gelatinous material	synthesized from basal side of an epithelial sheet. 
It creates a cell-free space into which cells can proliferate and migrate, perform the diffusion of nutrients, metabolites,...
It is able to resist compressive forces and acts as space filling in embryogenesis.
-	degraded by hyaluronidase-wound healing
-	simplest GAG
-	very long chain length (25000 disaccharide units)
-	lacks sulfated sugars
-	usually not attached to protein (used as filler)
 
 
Collagen assembly:
 
Regeneration of amputated limb: in amphibians and fish it happens also in adults. It’s super difficult, you have to regenerate a complex structure! How? There are specific tissue regeneration mechanism
In adults of Homo Sapiens the liver regenerates spontaneously. We have little capacity to regenerate tendons and ligaments. Stem cells can differentiate in something
 

\begin{figure}[h]
\includegraphics[width=1\textwidth]{ECMfunction.png}
\caption{\label{fig:ECM} ECM functionalization}
\end{figure}
