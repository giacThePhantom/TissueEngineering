\graphicspath{{chapters/06/images/}}
\chapter{Hydrogels}


\section{Introduction}
Hydrogels were the first biomaterials rationally designed for human use, especially for soft tissue engineering.

	\subsection{Composition}
	They are a particular class of materials, exhibiting both solid-like and liquid-like properties, consisting of hydrophilic polymer chains and water that occupies the interstitial spaces or pores that are defined in the 3D network constituted by the cross-linked polymeric chains.

	\subsection{Water content}
	Water is the major constituent of hydrogels and could reach more than $90\%$ of the weigh of the material.
	Its quantity is dependent on the degree of hydrophilicity of the polymer chains.

	\subsection{Seeding with cells}
	Cells can be added inside the hydrogel before or after gelatization.
	In the first case, one must make sure that the gelatization itself won’t damage the cells.
	In the second case, porosity must allow for uniform cellular colonization.

\section{Characteristics}
Hydrogels’ characteristics can be modulated according to three main variables:

\begin{multicols}{3}
	\begin{itemize}
		\item Chain composition.
		\item Cross-linking nature.
		\item Network nature.
	\end{itemize}
\end{multicols}


	\subsection{Chain composition}
	Hydrogels can be natural (usually polysaccharides) or synthetic polymers.
	The main subvariables are:

	\begin{multicols}{2}
		\begin{itemize}
			\item Chain length.
			\item Degree of hydrophilicity.
			\item Presence of ligands recognizable by cellular receptors.
				The utility of synthetic polymers is that they can be functionalized with these kinds of ligands to allow cell adhesion, but degradability will become a problem.
		\end{itemize}
	\end{multicols}

	\subsection{Cross-linking nature}
	With cross linking or gelatization nature is intended the strategy employed to connect the polymeric chains

		\subsubsection{Chemical cross-linking}
		The polymer chains are covalently linked.
		This linkage can be obtained in an enzymatic way if enzymes capable of interacting with the chosen polymers exist.
		The enzymatic cross-linking is useful because it guarantees the degradability of the scaffold via host enzymes or in a non-enzymatic way, exploiting specific reactions that may require specific reagents.

		\subsubsection{Physical cross-linking}
		In physical cross-linking polymer chains are held together by molecular entanglements, a temporary spatial constraints and secondary forces such as ionic, hydrophobic, and hydrogen bonds.
		Chemically crosslinked hydrogels always present some degree of physical interaction as well, but the contrary, of course, does not occur.
		Physical crosslinking can be obtained via thermal treatment, pH changes and treatment with organic solvents.

	\subsection{Network nature}
	Network nature: it’s the overall 3D structure defined by the crosslinked polymers.
	Its main characteristic are the number and the dimension of the interstitial spaces or pores, influenced by polymer chains length and by the number of crosslinkings.
	Pores number and dimension, together with the degree of hydrophilicity of the polymer chains, will determine the maximum amount of water hosted by the hydrogel.
	This in turn will impact the swelling capacity, one of the main characteristics of these kinds of scaffolds.

\section{Other uses of hydrogels}

	\subsection{Hydrogels as cells carriers}
	Hydrogels can also be designed with the only goal of carrying modded cells to a particular region of the body: in that case non biocompatible, non biorecognizable synthetic polymers can be used to avoid premature hydrogel degradation and unwanted interaction with the cargo and to carry the cells to the targeted location.

	\subsection{Hydrogels as drug releasing systems}
	Hydrogels can be used as drug release systems: they can function as surrogate matrices to carry modified cells producing a specific therapeutic agent.
	In order to protect these cells from the host’s immune system, the hydrogel itself must be coated by a semipermeable membrane, that must allow the entrance of nutrients and growth factors in the hydrogel and the exiting of the therapeutic products.
