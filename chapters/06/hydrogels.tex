\graphicspath{{chapters/06/images/}}
\chapter{Hydrogels}
Hydrogels were the first biomaterials rationally designed for human use, especially for soft tissue engineering.
They are a particular class of materials, exhibiting both solid-like and liquid-like properties, consisting of hydrophilic polymer chains and water, that occupies the interstitial spaces (pores) that are defined in the 3D network constituted by the cross-linked polymeric chains.
\\
\\
\noindent
Water is the major constituent of hydrogels and could reach more than 90 percent of the weigh of the material - although its quantity is dependent on the degree of hydrophilicity of the polymer chains.
Cells can be added inside the hydrogel before or after gelation.
In the first case, one must make sure that the gelation itself won’t damage the cells.
In the second case, porosity must allow for uniform cellular colonization.
\\
\\
\noindent
Hydrogels’ characteristics can be modulated according to three main variables:
\begin{itemize}
\item Chain composition: they can be natural (usually polysaccharides) or synthetic polymers. The main subvariables are chain length, degree of hydrophilicity and presence of ligands recognizable by cellular receptors (hydrogels must be biorecognizable). The utility of synthetic polymers is that they can be functionalized with these kinds of ligands to allow cell adhesion, but degradability will become a problem.
\item Cross-linking nature: strategy employed to connect the polymeric chains. This act is also called gelation. This could be:
	\begin{itemize}
	\item Chemical crosslinking: the polymer chains are covalently linked.
This linkage can be obtained in an enzymatic way (if enzymes capable of interacting with the chosen polymers exist).
The enzymatic crosslinking is also useful because it guarantees the degradability of the scaffold via host enzymes or in a non - enzymatic way, exploiting specific reactions that may require specific reagents.
	\item Physical crosslinking: the polymer chains are held together by molecular entanglements (temporary spatial constraints) and/or secondary forces such as ionic, hydrophobic, and hydrogen bonds. Chemically crosslinked hydrogels always present some degree of physical interaction as well, but the contrary, of course, does not occur.
Physical crosslinking can be obtained via thermal treatment, pH changes and treatment with organic solvents.
	\end{itemize}
\item Network nature: it’s the overall 3D structure defined by the crosslinked polymers.
Its main characteristic is the number and the dimension of the interstitial spaces (pores), influenced by polymer chains length and by the number of crosslinkings.
Pores number and dimension, together with the degree of hydrophilicity of the polymer chains, will determine the maximum amount of water hosted by the hydrogel.
This in turn will impact the swelling capacity, one of the main characteristics of these kinds of scaffolds.
\end{itemize}
\noindent
Hydrogels can also be designed with the only goal of carrying modded cells to a particular region of the body: in that case non biocompatible, non biorecognizable synthetic polymers can be used to avoid premature hydrogel degradation and unwanted interaction with the cargo and to carry the cells to the targeted location.
\\
\\
\noindent
Hydrogels can also be used to design drug releasing systems: they can function as surrogate matrices to carry modified cells producing a specific therapeutic agent.
In order to protect these cells from the host’s immune system, the hydrogel itself must be coated by a semipermeable membrane, that must allow the entrance of nutrients and growth factors in the hydrogel and the exiting of the therapeutic products.
