\chapter{Bioprinting} 

\section{Bioink composition}
Bioinks are inks comprised of cells and other biomaterials and are defined by their printability and cytocompatibility. Their printability influences shape fidelity and mechanical stability, whereas cytocompatibility dictates cell viability, migration, proliferation, differentiation and subsequent tissue formation. Bioink properties are chosen to compliment bioprinter type as well as bioprinting approach according to the desired tissue. 

\section{Bioprinting approaches}
\begin{itemize}
\item scaffold-based: use biomaterials to create a temporary structure that supports cell attachment, proliferation, and subsequent tissue formation. It is more economical and scalable due to its lower cell density requirements, and it provides higher resolution when compared to scaffold-free techniques. However, the presence of exogenous scaffolds can reduce cell-to-cell interactions and degrade into toxic byproducts over time.
\item scaffold-free: depends on autonomous self-assembly of the tissue as it develops. This approach attempts to replicate embryonic environmental and structural development by enabling cells to assemble autonomously. Prefabricated multicellular building blocks such as cell pellets, spheroids, or tissue strands are utilized to generate 3D constructs.  This cell friendly approach avoids the use of exogenous material, ultimately reducing toxicity, improving cell viability, increasing cell-to-cell interactions and reducing the length of post-bioprinting maturation when compared to scaffold-based bioprinting. However, it requires higher cell densities limiting printer selection, possesses low scalability, and lacks mechanical integrity due to the absence of scaffold or physical support 
\end{itemize}

\section{Printing technologies}
\begin{itemize}
\item inkjet: surface tension holds the bioink at the nozzle of the printer and several strategies are used to force droplets out in a controlled fashion e.g. thermal inkjet printers apply bursts of 200°C energy for ejecting the biomaterial in a dropwise fashion.
\item extrusion based:  capability of printing cell dense, viscous bioinks, using one of three main mechanisms to continuously force viscous biomaterial out of the nozzle in a controlled manner. Despite the possibility of nozzle clogging and low printing resolution, the main advantage of this scalable technology is its ability to print high, biologically relevant, viscosities allowing the use of scaffold-free spheroid bioinks.
\begin{itemize}
	\item pneumatic mechanism: applies air pressure to the surface of the bioinks. 
	\item mechanical mechanism: applies mechanical force to the surface of the bioink using a piston,
	\item screw-based mechanism: applies a rotational force to continuously extrude bioinks. 
\end{itemize}
\item light based: stereolithography bioprinting uses a pool of liquified cell laden biopolymer that is photoactivated by UV light. Precise movement of the UV light by a computer causes macromolecules to crosslink in a highly controlled manner and stimulates the development of tissue architecture. This nozzle free approach avoids the issue of nozzle clogging, but the bioinks are limited because they are required to have the ability to photo polymerize and there is a risk of damaging the cell DNA due to UV light exposure.
\end{itemize}

\section{Challenges in printing}
\begin{itemize}
\item injectability: printing fidelity generally increases with increasing viscosity, but printheads can only bear a certain amount.
\item shape retention
\end{itemize}

\section{Volumetric Bioprinting of Complex Living-Tissue Constructs within Seconds}
The current paradigm in bioprinting relies on the additive layer-by-layer deposition and assembly of repetitive building blocks, typically cell-laden hydrogel fibers or voxels, single cells, or cellular aggregates.  Volumetric bioprinting permits the creation of geometrically complex, centimeter-scale constructs at an unprecedented printing velocity, opening new avenues for upscaling the  production of hydrogel-based constructs and for their application in tissue engineering, regenerative medicine, and soft robotics.

\section{In situ bioprinting}
In situ bioprinting (also sometimes referred to as ‘‘in vivo” bioprinting) can be defined as direct printing of bioinks to create or repair living tissues or organs at a defect site in a clinical setting. The site of in situ printing could be the anatomical location, which needs regeneration in the body.
The two main in situ bioprinting approaches are:
\begin{itemize}
\item robotic arm: the arm comprises a 3-axis movable bioprinting unit, which aims to perform real-time printing to fabricate or repair the tissues and organs with the maximum possible hierarchical and physiological resemblance with the native structures
\item handheld: involves a highly portable device with a bioprinting unit, which allows the deposition of biomaterials with a specific living cell in a direct-write fashion.
\end{itemize}

Challenges:
\begin{itemize}
\item cell source, biomaterial selection and post-printing maturation of tissue
\item cost
\item intellectual property rights and ethical dilemma
\item need to develop more advanced bioinks, printing methods and real-time monitoring devices
\end{itemize}

